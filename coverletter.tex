Dear Editor, 

We are submitting a Full Paper, titled "Training a Gaming Agent on Brainwaves" which we believe is within the scope of IEEE Transactions on Games.  This article encompass work performed at CiC Laboratory of the ITBA University in Buenos Aires, Argentina, in the context of research on Brain Computer Interfaces and Artificial Intelligence.  This is original work, which has not been submitted for publication elsewhere.

This work details an experiment performed by 8 subjects that watched a gaming agent playing a simple game. Subjects used a wireless EEG device to capture brainwaves.  We detected Error-Potentials (ErrP) which are signal components that can be found on EEG traces that are triggered when the subject subjectively determines an erroneous outcome.  By classifying the signals we derived rewards penalizing the agent movements that the person considered as an error.  Using that information, the gaming agent was trained using Q-Learning, a Reinforcement Learning algorithm.  We show that the agent learns a policy that improves its operational performance.

We believe this is an interesting work that explore a novel interaction scheme, where the interaction between a game and a person is not explicit.  Instead, the information is sent surreptitiously, or extracted from the person to derive information for the gaming agent.  This is a challenging area of research and our work shows how to implement a simple game that triggered this action potential from users.

Additionally, for the sake of reproducibility we used open source libraries for our software implementation and uploaded the obtained dataset to the IEEE Dataport initiative. 

Yours truly
Rodrigo Ramele
ITBA University 
Argentina